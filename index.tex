% !TeX root = RJwrapper.tex
\title{Gathering Bibliometric Information using Scopus using rscopus}
\author{by John Muschelli}

\maketitle

\abstract{%
We demonstrate how to download author and affiliation using the
\texttt{rscopus} package, interacing with the Elsevier Scopus API. We
show histograms of the number of citations from an author, as well as
calculating citation metrics with the output.
}

% Any extra LaTeX you need in the preamble

\hypertarget{introduction}{%
\subsection{Introduction}\label{introduction}}

We would like to gather information about publications, authors, and
institutions with respect to published research. Scopus a repository of
information about scientific articles and books, which includes
information about authors, citations, and abstracts of these pieces of
literature. Scopus claims to have the largest database of this
information. Therefore, providing users an interface to this repository
should be worthwhile.

One common task for researchers is to keep the curriculum vitae (CV) up
to date. That requires having up to date information on the papers
published and under submission. Keeping track of these papers can be
tedious and automated solutions could exist, but are not widely used
(THINK OF SERVICES THAT TRY TO do this). One conern is always to be
missing certain crucial papers in your CV. Although \pkg{rscopus} does
not provide these tools specifically, it can be used to consistently
cross-reference information about publications with a CV.

Additionally on CVs, it is useful to present information of the impact
of a paper on the CV. This can be done by highlighting certain pieces of
information, such is done in NIH biosketches, or ranking them based on
some metric. The metric commonly used is the number of citations. Also,
information about the journal impact factor may be taken into account.
We do not note that these are particularly good metrics or metrics that
reflect true impact, but are simply those that we have seen used in
practice.

\pkg{rscopus} allows you to interface with Scopus APIs and gather
information about authors, affiliations, articles, and abstracts. The
\pkg{bibliometrix} package provides a level of integration that is
useful for using multiple packages that deal with bibliometric data,
incorporating functionality from \pkg{rscopus}. The \pkg{bibliometrix}
package also enables users to analyze data from ISI Web of Knowledge
(WoK) and PubMed. Web of Knowledge is one competitor to Scopus, but
\pkg{bibliometrix} does not have an interface to the WoK API; therefore
data must be manually exported from the site into \texttt{R}. Additional
access to the web of Science API would be useful and has been
implemented in a GitHub package \pkg{rwos}
(\url{https://github.com/juba/rwos}), but is not on CRAN.

Moreover, other packages such as \pkg{scholar} and \pkg{gcite} can
provide interfaces to the Google Scholar citation information. Using
these in combination with \pkg{rscopus} can more likely guarantee
complete information.

As compared to Google Scholar, WoK and Scopus information is
\textbf{more curated} (is that true?).

As many pieces of academic promotion are related citation metrics or
impact, these metrics can have real-life implications.

Here we will present a few examples of how to use the \pkg{rscopus}
package, and present a simple descriptive analysis of citations and
citation indices from people within the same field.

Scopus has a number of APIs available
(\url{https://dev.elsevier.com/sc_apis.html}). We will distinguish
between 2 types of APIs: search APIs and retrieval APIs. The retrieval
APIs require unique indentifiers, such as an author ID or affiliation
ID, to retrieve information. As these identifiers are not commonly
known, we will use the search APIs to search on other criteria to obtain
these identifiers. We will focus on gathering information about authors,
affiliations, and citations.

\hypertarget{api-key}{%
\subsection{API Key}\label{api-key}}

Before using the package, one must obtain an access key to the API from
\url{https://dev.elsevier.com/sc_apis.html} with the following steps:

\begin{enumerate}
\def\labelenumi{\arabic{enumi}.}
\tightlist
\item
  Go to \url{https://dev.elsevier.com/user/login}. Login or create a
  free account.
\item
  Click ``Create API Key''. Put in a label, such as
  \texttt{rscopus\ key}. Add a website. \url{http://example.com} is fine
  if you do not have a site.
\item
  \textbf{Read} and agree to the terms of service if you do indeed
  agree.
\item
  Add \texttt{Elsevier\_API\ =\ "API\ KEY\ GOES\ HERE"} to
  \texttt{\textasciitilde{}/.Renviron} file, or add
  \texttt{export\ Elsevier\_API=API\ KEY\ GOES\ HERE} to your
  \texttt{\textasciitilde{}/.bash\_profile}.
\end{enumerate}

Alternatively, you you can either set the API key using
\texttt{rscopus::set\_api\_key} or by
\texttt{options("elsevier\_api\_key"\ =\ api\_key)}. You can access the
API key using \texttt{rscopus::get\_api\_key}.

You should be able to test out the API key using the
\href{https://dev.elsevier.com/scopus.html}{interactive Scopus APIs}.

\hypertarget{a-note-about-api-keys-and-ip-addresses}{%
\subsubsection{A note about API keys and IP
addresses}\label{a-note-about-api-keys-and-ip-addresses}}

The API Key is bound to a set of IP addresses, usually bound to your
institution or organization (see
\url{https://dev.elsevier.com/tecdoc_api_authentication.html}).
Therefore, if you are using \pkg{rscopus} for a Shiny application, you
must host the Shiny application from the institution/organization
servers in some way. Also, you cannot access the Scopus API with this
key if you are offsite and must VPN into the server or use a computing
cluster with an institution IP.

\hypertarget{use-cases}{%
\subsection{Use cases}\label{use-cases}}

\hypertarget{processing-author-names-to-ids}{%
\subsubsection{Processing author names to
IDs}\label{processing-author-names-to-ids}}

Researchers commonly would like to gather information about a set of
authors. Most times the authors are the given by first and last names or
initials; additional information such as affiliation may be available.
Scopus provides unique identifier for authors (\texttt{au\_id}) or
affiliations (\texttt{affil\_id}). In many cases with the API, you will
specify the author identifier (\texttt{au\_id}) instead of a first and
last name, as there may be many authors with the same name. This
identifier is unique to this author, though curation errors do happen
and someone may have 2 unique identifiers. These identifiers can be
merged by request on the Scopus website. In order to get the identifier
from Scopus, you can search using a first and last name using the
\texttt{process\_author\_name} command. For example, let us try to
idnetify the author ID for John Muschelli:

\begin{Schunk}
\begin{Sinput}
library(rscopus)
auth_info = process_author_name(last_name = "Muschelli", first_name = "John")
\end{Sinput}
\begin{Soutput}
       auth_name       au_id affil_id
1 John Muschelli 40462056100 60006183
                                       affil_name
1 Johns Hopkins Bloomberg School of Public Health
\end{Soutput}
\begin{Sinput}
auth_info
\end{Sinput}
\begin{Soutput}
$first_name
[1] "John"

$last_name
[1] "Muschelli"

$au_id
[1] "40462056100"
\end{Soutput}
\end{Schunk}

The output is a simple list of first and last name with an author ID.
The function chooses the first author found, which may be useful if the
author name is somewhat unique.

\hypertarget{retrieving-author-citation-data}{%
\subsubsection{Retrieving author citation
data}\label{retrieving-author-citation-data}}

In order to get data about author papers and citations, the
\texttt{author\_data} function will retrieve this information:

\begin{Schunk}
\begin{Sinput}
jm = author_data(last_name = "Muschelli", first_name = "John")
\end{Sinput}
\begin{Soutput}
       auth_name       au_id affil_id
1 John Muschelli 40462056100 60006183
                                       affil_name
1 Johns Hopkins Bloomberg School of Public Health
list(query = "AU-ID(40462056100)", count = 25, start = 0, view = "COMPLETE", 
    facets = "subjarea(sort=fd,count=350)")
$query
[1] "AU-ID(40462056100)"

$count
[1] 25

$start
[1] 0

$view
[1] "COMPLETE"

$facets
[1] "subjarea(sort=fd,count=350)"

Response [https://api.elsevier.com/content/search/scopus?query=AU-ID%2840462056100%29&count=25&start=0&view=COMPLETE&facets=subjarea%28sort%3Dfd%2Ccount%3D350%29]
  Date: 2018-10-15 18:45
  Status: 200
  Content-Type: application/json;charset=UTF-8
  Size: 257 kB


  |                                                                       
  |=================================================================| 100%
\end{Soutput}
\begin{Sinput}
names(jm)
\end{Sinput}
\begin{Soutput}
[1] "entries"   "df"        "full_data"
\end{Soutput}
\end{Schunk}

We see the output is a list of the converted \texttt{entries} from the
Scopus API, a \texttt{data.frame} of the results for citations, and a
list named \texttt{full\_data}. The \texttt{data.frame} \texttt{df} has
the information many users wish to retrieve, which is information about
the author's documents and citations:

\begin{Schunk}
\begin{Sinput}
head(jm$df[, c("dc:identifier", "eid", "dc:title", "dc:creator", 
"prism:publicationName", "prism:doi", "citedby-count",  "pii", "pubmed-id")])
\end{Sinput}
\begin{Soutput}
          dc:identifier                eid
1 SCOPUS_ID:85053246791 2-s2.0-85053246791
2 SCOPUS_ID:85043338865 2-s2.0-85043338865
3 SCOPUS_ID:85047750078 2-s2.0-85047750078
4 SCOPUS_ID:85028874240 2-s2.0-85028874240
5 SCOPUS_ID:85050271095 2-s2.0-85050271095
6 SCOPUS_ID:85009266881 2-s2.0-85009266881
                                                                                                                                                                     dc:title
1                                                        Objective Evaluation of Multiple Sclerosis Lesion Segmentation using a Data Management and Processing Infrastructure
2                                                                        MIMoSA: An Automated Method for Intermodal Segmentation Analysis of Multiple Sclerosis Brain Lesions
3                       Radiomic subtyping improves disease stratification beyond key molecular, clinical, and standard imaging characteristics in patients with glioblastoma
4                                    Feasibility of Coping Effectiveness Training for Caregivers of Children with Autism Spectrum Disorder: a Genetic Counseling Intervention
5                                                                                     Freesurfer: Connecting the Freesurfer software with R [version 1; referees: 2 approved]
6 Thrombolytic removal of intraventricular haemorrhage in treatment of severe stroke: results of the randomised, multicentre, multiregion, placebo-controlled CLEAR III trial
          dc:creator         prism:publicationName
1       Commowick O.            Scientific Reports
2       Valcarcel A.       Journal of Neuroimaging
3   Kickingereder P.                Neuro-Oncology
4 Haakonsen Smith C. Journal of Genetic Counseling
5       Muschelli J.                 F1000Research
6          Hanley D.                    The Lancet
                       prism:doi citedby-count               pii pubmed-id
1     10.1038/s41598-018-31911-7             0              <NA>      <NA>
2              10.1111/jon.12506             0              <NA>      <NA>
3          10.1093/neuonc/nox188             3              <NA>      <NA>
4      10.1007/s10897-017-0144-1             2              <NA>      <NA>
5 10.12688/f1000research.14361.1             0              <NA>      <NA>
6  10.1016/S0140-6736(16)32410-2            46 S0140673616324102  28081952
\end{Soutput}
\end{Schunk}

We see that the \texttt{full\_data} has this \texttt{df} inside it, with
other \texttt{data.frame}s:

\begin{Schunk}
\begin{Sinput}
names(jm$full_data)
\end{Sinput}
\begin{Soutput}
[1] "df"             "affiliation"    "author"         "openaccessFlag"
\end{Soutput}
\end{Schunk}

These additional \texttt{data.frame}s can have additional information
about co-author affiliations or co-author information. This information
may be useful for creating network graphs. For example, to get all
authors from all the papers, you can use the \texttt{author} element
from \texttt{full\_data}:

\begin{Schunk}
\begin{Sinput}
head(jm$full_data$author)
\end{Sinput}
\begin{Soutput}
  @_fa @seq                                                    author-url
1 true    1  https://api.elsevier.com/content/author/author_id/8431704700
2 true    2 https://api.elsevier.com/content/author/author_id/57203861434
3 true    3 https://api.elsevier.com/content/author/author_id/57199507814
4 true    4 https://api.elsevier.com/content/author/author_id/57197801981
5 true    5 https://api.elsevier.com/content/author/author_id/57203867656
6 true    6 https://api.elsevier.com/content/author/author_id/57203864793
       authid     authname   surname given-name initials afid.@_fa
1  8431704700 Commowick O. Commowick    Olivier       O.      true
2 57203861434    Istace A.    Istace     Audrey       A.      true
3 57199507814      Kain M.      Kain    Michaël       M.      true
4 57197801981   Laurent B.   Laurent   Baptiste       B.      true
5 57203867656     Leray F.     Leray    Florent       F.      true
6 57203864793     Simon M.     Simon    Mathieu       M.      true
    afid.$ entry_number
1 60030553            1
2 60001780            1
3 60030553            1
4 60105610            1
5 60030553            1
6 60030553            1
\end{Soutput}
\end{Schunk}

The column \texttt{entry\_number} should merge with the
\texttt{data.frame} of citations, as well as the information about
author affiliations, which is located in the \texttt{affiliation}
\texttt{data.frame} from \texttt{full\_data}:

\begin{Schunk}
\begin{Sinput}
head(jm$full_data$affiliation)
\end{Sinput}
\begin{Soutput}
  @_fa
1 true
2 true
3 true
4 true
5 true
6 true
                                                       affiliation-url
1 https://api.elsevier.com/content/affiliation/affiliation_id/60030553
2 https://api.elsevier.com/content/affiliation/affiliation_id/60001780
3 https://api.elsevier.com/content/affiliation/affiliation_id/60105610
4 https://api.elsevier.com/content/affiliation/affiliation_id/60062760
5 https://api.elsevier.com/content/affiliation/affiliation_id/60028893
6 https://api.elsevier.com/content/affiliation/affiliation_id/60028893
      afid
1 60030553
2 60001780
3 60105610
4 60062760
5 60028893
6 60028893
                                                                affilname
1                                                  Universite de Rennes 1
2                                             Centre Hospitalier Lyon-Sud
3                     Laboratoire de Traitement de l'Information Medicale
4 Centre de Recherche en Acquisition et Traitement d'Images pour la Sante
5                              Centre Hospitalier Universitaire de Rennes
6                              Centre Hospitalier Universitaire de Rennes
  affiliation-city affiliation-country entry_number name-variant.@_fa
1           Rennes              France            1              <NA>
2             Lyon              France            1              <NA>
3            Brest              France            1              <NA>
4     Villeurbanne              France            1              <NA>
5           Rennes              France            1              <NA>
6           Rennes              France            1              <NA>
  name-variant.$
1           <NA>
2           <NA>
3           <NA>
4           <NA>
5           <NA>
6           <NA>
\end{Soutput}
\end{Schunk}

This information is rich for understanding information about an author's
publication record, how many citations are recorded for a specific
article, which journals have been published in, and who has co-authored
publications with an author.

\hypertarget{calculating-author-indices}{%
\subsubsection{Calculating author
indices}\label{calculating-author-indices}}

With the data from the \texttt{author\_data} output, we can calculate
the overall H-index (CITE) as follows:

\begin{Schunk}
\begin{Sinput}
library(dplyr)
h_data = jm$df %>% 
  mutate(citations = as.numeric(`citedby-count`)) %>% 
  arrange(-citations) %>% 
  mutate(paper = 1,
         n_papers = cumsum(paper))
h_index = max(which(h_data$citations >= h_data$n_papers))
h_index
\end{Sinput}
\begin{Soutput}
[1] 15
\end{Soutput}
\end{Schunk}

Using \pkg{ggplot2}, we can also visually show the H-index computation,
where we plot the number of citations versus the number of papers
(cumulatively) along with the X-Y line:

\begin{Schunk}
\begin{Sinput}
library(ggplot2)
h_data %>% 
  ggplot(aes(x = n_papers, y = citations)) + 
  geom_point() + geom_abline(slope = 1, intercept = 0) + 
  geom_hline(yintercept = h_index, color = "red")
\end{Sinput}

\includegraphics{index_files/figure-latex/unnamed-chunk-7-1} \end{Schunk}

Additional indices can be created from the dat.

\hypertarget{retrieving-information-about-an-author}{%
\subsubsection{Retrieving information about an
author}\label{retrieving-information-about-an-author}}

In \texttt{process\_author\_name}, we demonstrated how to get
information form an author with a relatively unique name. If this is not
the case, the \texttt{get\_complete\_author\_info}, which powers
\texttt{process\_author\_name}, can present more results. In order to
retrieve author IDs from first and last names, the
\texttt{get\_complete\_author\_info} can be used. Here we search for
authors with the last name West and first initial M:

\begin{Schunk}
\begin{Sinput}
last_name = "West"
first_name = "M"
auth_info_list = get_complete_author_info(
  last_name = last_name, 
  first_name = first_name)
class(auth_info_list)
\end{Sinput}
\begin{Soutput}
[1] "list"
\end{Soutput}
\begin{Sinput}
names(auth_info_list)
\end{Sinput}
\begin{Soutput}
[1] "get_statement" "content"      
\end{Soutput}
\end{Schunk}

We see here, which is common in in some low-level functions returned
from the API, the output is a list with elements
\texttt{get\_statement}, which returns an object of class
\texttt{response} (from the \pkg{pttr} package), and \texttt{content},
which is the content from the response. Most times, the \texttt{content}
is of interest, but failed requrests may be explored with the
\texttt{get\_statement} output.

In many \texttt{content} elements returned from the API, there are
elements of the list named \texttt{entries} or \texttt{entry}. The
low-level function \texttt{gen\_entries\_to\_df} attempts to coerce this
list into a standard \texttt{data.frame} for more usability, but may not
perform perfectly as lists from JSON cannot always be directly coerced
into a rectangular format. For example, here we will convert that output
into a \texttt{data.frame}:

\begin{Schunk}
\begin{Sinput}
coerced = gen_entries_to_df(auth_info_list$content$`search-results`$entry)
names(coerced)
\end{Sinput}
\begin{Soutput}
[1] "df"           "name-variant" "subject-area"
\end{Soutput}
\begin{Sinput}
head(coerced$df)
\end{Sinput}
\begin{Soutput}
  @_fa                                                     prism:url
1 true https://api.elsevier.com/content/author/author_id/35480328200
2 true https://api.elsevier.com/content/author/author_id/35419377800
3 true  https://api.elsevier.com/content/author/author_id/7003392768
4 true  https://api.elsevier.com/content/author/author_id/7402395730
5 true  https://api.elsevier.com/content/author/author_id/7402068812
6 true  https://api.elsevier.com/content/author/author_id/7401998578
          dc:identifier                eid               orcid
1 AUTHOR_ID:35480328200 9-s2.0-35480328200 0000-0002-0839-3449
2 AUTHOR_ID:35419377800 9-s2.0-35419377800                <NA>
3  AUTHOR_ID:7003392768  9-s2.0-7003392768                <NA>
4  AUTHOR_ID:7402395730  9-s2.0-7402395730                <NA>
5  AUTHOR_ID:7402068812  9-s2.0-7402068812                <NA>
6  AUTHOR_ID:7401998578  9-s2.0-7401998578                <NA>
  preferred-name.surname preferred-name.given-name preferred-name.initials
1                   West                 Catharine                      C.
2                   West                Malcolm J.                    M.J.
3            Diener-West                     Marie                      M.
4                   West                 Robert M.                    R.M.
5                   West           Michael Abigail                    M.A.
6                   West                  David M.                    D.M.
  document-count
1            269
2            191
3            186
4            166
5            161
6            157
                                            affiliation-current.affiliation-url
1          https://api.elsevier.com/content/affiliation/affiliation_id/60003771
2          https://api.elsevier.com/content/affiliation/affiliation_id/60019870
3          https://api.elsevier.com/content/affiliation/affiliation_id/60006183
4          https://api.elsevier.com/content/affiliation/affiliation_id/60012070
5 https://api.elsevier.com/content/affiliation/affiliation_id/60012018 60023691
6          https://api.elsevier.com/content/affiliation/affiliation_id/60008221
  affiliation-current.affiliation-id
1                           60003771
2                           60019870
3                           60006183
4                           60012070
5                  60012018 60023691
6                           60008221
             affiliation-current.affiliation-name
1                        University of Manchester
2                James Cook University, Australia
3 Johns Hopkins Bloomberg School of Public Health
4                             University of Leeds
5         University of Pittsburgh Medical Center
6                               Massey University
  affiliation-current.affiliation-city
1                           Manchester
2                           Townsville
3                            Baltimore
4                                Leeds
5             Pittsburgh San Francisco
6                     Palmerston North
  affiliation-current.affiliation-country entry_number
1                          United Kingdom            1
2                               Australia            2
3                           United States            3
4                          United Kingdom            4
5             United States United States            5
6                             New Zealand            6
\end{Soutput}
\end{Schunk}

We see this has information about the multiple authors returned, along
with names, variations on those names, number of documents, and
affiliations. We can then extract the author ID we want from this
\texttt{data.frame}. This process is wrapped in the
\texttt{get\_author\_info}:

\begin{Schunk}
\begin{Sinput}
auth_info_df = get_author_info(last_name = last_name, 
                              first_name = first_name)
head(auth_info_df)
\end{Sinput}
\begin{Soutput}
             auth_name       au_id          affil_id
1       Catharine West 35480328200          60003771
2      Malcolm J. West 35419377800          60019870
3    Marie Diener-West  7003392768          60006183
4       Robert M. West  7402395730          60012070
5 Michael Abigail West  7402068812 60012018 60023691
6        David M. West  7401998578          60008221
                                       affil_name
1                        University of Manchester
2                James Cook University, Australia
3 Johns Hopkins Bloomberg School of Public Health
4                             University of Leeds
5         University of Pittsburgh Medical Center
6                               Massey University
\end{Soutput}
\end{Schunk}

but we should note this information is condensed and a subset that is
available from \texttt{get\_complete\_author\_info}.

If we now have an affiliation ID, such as \texttt{60006183} for the
Johns Hopkins Bloomberg School of Public Health, we can pass this to
\texttt{get\_author\_info} or \texttt{process\_author\_name}:

\begin{Schunk}
\begin{Sinput}
spec_affil = get_author_info(
  last_name = last_name, 
  first_name = first_name,
  affil_id = 60006183)
spec_affil
\end{Sinput}
\begin{Soutput}
          auth_name      au_id affil_id
1 Marie Diener-West 7003392768 60006183
                                       affil_name
1 Johns Hopkins Bloomberg School of Public Health
\end{Soutput}
\end{Schunk}

\hypertarget{retrieving-summary-information-about-an-author}{%
\subsubsection{Retrieving summary information about an
author}\label{retrieving-summary-information-about-an-author}}

The \texttt{author\_retrieval} function can gather summary information
about an author using the author identifier or name.

\begin{Schunk}
\begin{Sinput}
author_info = author_retrieval(last_name = "Muschelli", first_name = "J")
\end{Sinput}
\begin{Soutput}
       auth_name       au_id affil_id
1 John Muschelli 40462056100 60006183
                                       affil_name
1 Johns Hopkins Bloomberg School of Public Health
\end{Soutput}
\begin{Sinput}
names(author_info$content)
\end{Sinput}
\begin{Soutput}
[1] "author-retrieval-response"
\end{Soutput}
\begin{Sinput}
class(author_info$content$`author-retrieval-response`)
\end{Sinput}
\begin{Soutput}
[1] "list"
\end{Soutput}
\end{Schunk}

We can use \texttt{gen\_entries\_to\_df} to convert this response into a
\texttt{data.frame}

\begin{Schunk}
\begin{Sinput}
gen_entries_to_df(author_info$content$`author-retrieval-response`)$df
\end{Sinput}
\begin{Soutput}
  @status @_fa
1   found true
                                            coredata.prism:url
1 http://api.elsevier.com/content/author/author_id/40462056100
  coredata.dc:identifier       coredata.eid      coredata.orcid
1  AUTHOR_ID:40462056100 9-s2.0-40462056100 0000-0001-6469-1750
  coredata.document-count coredata.cited-by-count coredata.citation-count
1                      36                     671                     788
                                                                             coredata.link.@href
1 https://www.scopus.com/authid/detail.uri?partnerID=HzOxMe3b&authorId=40462056100&origin=inward
  coredata.link.@rel coredata.link.@_fa preferred-name.surname
1      scopus-author               true              Muschelli
  preferred-name.given-name preferred-name.initials
1                      John                      J.
             affiliation-current.affiliation-name
1 Johns Hopkins Bloomberg School of Public Health
  affiliation-current.affiliation-city
1                            Baltimore
  affiliation-current.affiliation-country publication-range.start
1                           United States                    2011
  publication-range.end entry_number
1                  2018            1
\end{Soutput}
\end{Schunk}

but this list typically only has one element, and may be easily
referenced using \texttt{\$} as a list.

\hypertarget{retrieving-information-about-multiple-authors}{%
\subsubsection{Retrieving information about multiple
authors}\label{retrieving-information-about-multiple-authors}}

In order to get information from multiple authors, one could loop over
author information, but this is inefficient for code and API calls. The
\texttt{complete\_multi\_author\_info} function can perform this
operation. One caveat is that it requires author identifiers and not
names. We can take the author IDs from \texttt{auth\_info\_df} to
retrieve information for all these authors:

\begin{Schunk}
\begin{Sinput}
all_author_info = complete_multi_author_info(au_id = auth_info_df$au_id)
names(all_author_info)
\end{Sinput}
\begin{Soutput}
[1] "get_statement" "content"       "au_id"        
\end{Soutput}
\end{Schunk}

This result is again a low-level output from the API. We can use the
\texttt{process\_complete\_multi\_author\_info} function to process this
into a more amenable solution:

\begin{Schunk}
\begin{Sinput}
processed = process_complete_multi_author_info(all_author_info)
head(names(processed))
\end{Sinput}
\begin{Soutput}
[1] "35480328200" "35419377800" "7003392768"  "7402395730"  "7402068812" 
[6] "7401998578" 
\end{Soutput}
\end{Schunk}

Now, each element is the author ID, which contains a list of
\texttt{data.frame}s. The \texttt{multi\_author\_info} will perform both
of these operations together. This result is still not ``tidy'' in many
respects, but parts can be combined using \texttt{dplyr} or
\texttt{purrr}:

\begin{Schunk}
\begin{Sinput}
journals = purrr:::map_df(processed, function(x) {
  x$journals
  }, .id = "au_id")
head(journals)
\end{Sinput}
\begin{Soutput}
        au_id type                                        sourcetitle
1 35480328200    j                                     Cancer Letters
2 35480328200    j   Journal of Cancer Research and Clinical Oncology
3 35480328200    j                              Nature Reviews Cancer
4 35480328200    j Annals of the Royal College of Surgeons of England
5 35480328200    j                                     Cancer Letters
6 35480328200    j                         PLoS Computational Biology
           sourcetitle-abbrev     issn
1                Cancer Lett. 03043835
2 J. Cancer Res. Clin. Oncol. 01715216
3            Nat. Rev. Cancer 1474175X
4   Ann. R. Coll. Surg. Engl. 00358843
5                Cancer Lett. 18727980
6          PLoS Comput. Biol. 1553734X
\end{Soutput}
\end{Schunk}

\hypertarget{citations-over-time}{%
\subsubsection{Citations over time}\label{citations-over-time}}

Some APIs from Elsevier are disabled by default (see
\url{https://dev.elsevier.com/api_key_settings.html}). Notably, the
Citations Overview API is disabled, which allows users to access
information about citations over time for articles of authors. This
information is particularly useful for creating bibliometric indices,
such as the \emph{h}-index (CITE). The \texttt{rscopus} package
interfaces with these APIs, but the API must be enabled for that
specific API key. On the Scopus website one can searching for authors,
select up to 15 authors, and then create a ``Citation Overview'', which
will give this citation information, which is in a CSV format. The
\texttt{rscopus} package provides a \texttt{read\_cto} function to read
in this data.

We also provide an example export from a single author:

\begin{Schunk}
\begin{Sinput}
file = system.file("extdata", "CTOExport.csv", package = "rscopus")
citations_over_time = rscopus::read_cto(file)
names(citations_over_time)
\end{Sinput}
\begin{Soutput}
[1] "data"               "year_columns"       "author_information"
\end{Soutput}
\end{Schunk}

The real information is in the \texttt{data} element of this list.\\
Here we present \texttt{short\_title}, first 3 (relevant) words of the
title, instead of the full document title for viewing purposes as titles
can be quite long.

\begin{Schunk}
\begin{Sinput}
yr_cols = citations_over_time$year_columns
citations_over_time = citations_over_time$data
citations_over_time = citations_over_time %>% 
  mutate(short_title = unique_title(`Document Title`))
head(citations_over_time[, c("short_title", yr_cols[1:5])])
\end{Sinput}
\begin{Soutput}
                            short_title <2008 2008 2009 2010 2011
1         Objective Evaluation Multiple     0    0    0    0    0
2              MIMoSA: Automated Method     0    0    0    0    0
3           Radiomic subtyping improves     0    0    0    0    0
4      Feasibility Coping Effectiveness     0    0    0    0    0
5     Freesurfer: Connecting Freesurfer     0    0    0    0    0
6 Thrombolytic removal intraventricular     0    0    0    0    0
\end{Soutput}
\end{Schunk}

In the citation overview, you must specify a range of years on Scopus,
with a maximum of 15 years. As many times this wide format is not what
you want to plot, a helper function \texttt{read\_cto\_long} will read
the data in long format, done by \pkg{tidyr} (CTIE). Here we use
\pkg{dplyr} to arrange the data by maximum number of citations:

\begin{Schunk}
\begin{Sinput}
library(dplyr)
long_cite = rscopus::read_cto_long(file)
long_cite = long_cite$data %>% 
  group_by(`Document Title`, year) %>% 
  summarize(citations = sum(citations), 
            `Publication Year` = unique(`Publication Year`)) %>% 
  mutate(short_title = unique_title(`Document Title`))
long_cite = long_cite %>% arrange(-citations, year, short_title)
head(long_cite[, c("short_title", "year", "citations")])
\end{Sinput}
\begin{Soutput}
# A tibble: 6 x 3
  short_title                           year  citations
  <chr>                                 <fct>     <int>
1 Minimally invasive surgery            2015         36
2 Minimally invasive surgery            2017         35
3 ISLES 2015 public                     2018         28
4 Large-scale radiomic profiling        2018         26
5 Thrombolytic removal intraventricular 2018         25
6 Minimally invasive surgery            2016         24
\end{Soutput}
\end{Schunk}

Thus, we have one record per year and article. Here we will plot the
cumulative citations per each paper over the years of publication and
label the top 3 cited papers:

\begin{Schunk}
\begin{Sinput}
# get cumulative sum
csum = long_cite %>% 
  mutate(citations = ifelse(is.na(citations), 0, citations)) %>% 
  arrange(`Document Title`, year) %>% 
  group_by(`Document Title`) %>% 
  mutate(citations = cumsum(citations))
# remove past and future with as.integer
csum = csum %>% 
  mutate(year = as.integer(as.character(year))) %>% 
  filter(!is.na(year)) %>% 
  filter(year >= `Publication Year`)
# grab last citations and top 3 papers
last_year = csum %>% 
  arrange(`Document Title`, year) %>% 
  group_by(`Document Title`) %>% 
  slice(n()) %>% 
  ungroup %>% arrange(-citations) %>% 
  head(3) 
g = ggplot(csum, 
           aes(x = year, y = citations, color = short_title  )) +
  xlim(c(2010, 2018)) + geom_line() + 
  # label the titles numbers for top 3
  geom_text(data = last_year, size = 3, aes(label = short_title), 
            nudge_x = -1, nudge_y = 5)
# don't want label for document title
g + guides(color = FALSE) + theme(text = element_text(size = 20))
\end{Sinput}

\includegraphics{index_files/figure-latex/unnamed-chunk-19-1} \end{Schunk}

\hypertarget{retrieving-affiliation-information}{%
\subsubsection{Retrieving Affiliation
Information}\label{retrieving-affiliation-information}}

In order to get information about an affiliation, the
\texttt{get\_affiliation\_info} can be used. Here we will look for the
pattern \texttt{Johns\ Hopkins}:

\begin{Schunk}
\begin{Sinput}
jhu_info = get_affiliation_info(affil_name = "Johns Hopkins")
head(jhu_info[, c("affil_id", "affil_name")])
\end{Sinput}
\begin{Soutput}
  affil_id                                              affil_name
1 60005248                                Johns Hopkins University
2 60001117                    The Johns Hopkins School of Medicine
3 60006183         Johns Hopkins Bloomberg School of Public Health
4 60001555                                  Johns Hopkins Hospital
5 60003443                      Johns Hopkins Medical Institutions
6 60022054 The Johns Hopkins University Applied Physics Laboratory
\end{Soutput}
\end{Schunk}

This function implicitly calls \texttt{affil\_search}, which searches
the affiliation information from Scopus. Additional information can be
extracted using \texttt{affil\_search}, but this typically includes a
large number of records as it searches all the documents. This
affiliation ID can be used to be more specific when searching authors or
documents.

\begin{Schunk}
\begin{Sinput}
eid_info = rscopus::article_retrieval(id = jm$df$eid[1], 
                                      identifier = "eid")
scopus_id = jm$df$`dc:identifier`[1]
scopus_id = sub("SCOPUS_ID:", "", scopus_id)
sc_info = rscopus::article_retrieval(id = scopus_id, identifier = "scopus_id")
doi_info = rscopus::article_retrieval(id = jm$df$`prism:doi`[1], identifier = "doi")
em_ret = embase_retrieval(id = jm$df$`prism:doi`[1], identifier = "doi")
em_content = em_ret$content$`abstracts-retrieval-response`
df = gen_entries_to_df(em_content)
res = generic_elsevier_api()
\end{Sinput}
\end{Schunk}

In some cases, one may have an article in mind and would like
information about the authors of that paper. In order to get the author
IDs from the paper identifier, one can use the
\texttt{abstract\_retrieval} function:

\begin{Schunk}
\begin{Sinput}
sc_id = sub("SCOPUS_ID:", "", jm$df$`dc:identifier`[1])
res = abstract_retrieval(id = sc_id, identifier = "scopus_id")
sc_info = res$content$`abstracts-retrieval-response`
sc_df = purrr::map_df(
  sc_info$authors[[1]],
  as.data.frame, 
  stringsAsFactors = FALSE,
  make.names = FALSE)
head(sc_df[, c("ce.given.name", "ce.initials", "X.auid")])
\end{Sinput}
\begin{Soutput}
  ce.given.name ce.initials      X.auid
1       Olivier          O.  8431704700
2        Audrey          A. 57203861434
3       Michaël          M. 57199507814
4      Baptiste          B. 57197801981
5       Florent          F. 57203867656
6       Mathieu          M. 57203864793
\end{Soutput}
\end{Schunk}

This information is located within the \texttt{author}
\texttt{data.frame} from the \texttt{full\_data} as well. As we took the
first entry from the Scopus identifier, we will subset the author data
by \texttt{entry\_number} \texttt{1} from the \texttt{author}
\texttt{data.frame}:

\begin{Schunk}
\begin{Sinput}
paper_author_info = jm$full_data$author
head(paper_author_info[paper_author_info$entry_number == 1,])
\end{Sinput}
\begin{Soutput}
  @_fa @seq                                                    author-url
1 true    1  https://api.elsevier.com/content/author/author_id/8431704700
2 true    2 https://api.elsevier.com/content/author/author_id/57203861434
3 true    3 https://api.elsevier.com/content/author/author_id/57199507814
4 true    4 https://api.elsevier.com/content/author/author_id/57197801981
5 true    5 https://api.elsevier.com/content/author/author_id/57203867656
6 true    6 https://api.elsevier.com/content/author/author_id/57203864793
       authid     authname   surname given-name initials afid.@_fa
1  8431704700 Commowick O. Commowick    Olivier       O.      true
2 57203861434    Istace A.    Istace     Audrey       A.      true
3 57199507814      Kain M.      Kain    Michaël       M.      true
4 57197801981   Laurent B.   Laurent   Baptiste       B.      true
5 57203867656     Leray F.     Leray    Florent       F.      true
6 57203864793     Simon M.     Simon    Mathieu       M.      true
    afid.$ entry_number
1 60030553            1
2 60001780            1
3 60030553            1
4 60105610            1
5 60030553            1
6 60030553            1
\end{Soutput}
\end{Schunk}

Thus, if we retrieve a single author's information, we can gather other
author IDs from this directly. If we have a specific paper, we can
retrieve author IDs from that paper information as well.


\address{%
John Muschelli\\
Department of Biostatistics, Johns Hopkins Bloomberg School of Public
Health\\
615 N Wolfe St Baltimore, MD 21205\\
}
\href{mailto:jmuschel@jhsph.edu}{\nolinkurl{jmuschel@jhsph.edu}}

